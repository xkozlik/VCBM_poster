%\documentclass[a4paper, 10pt, conference]{ieeeconf}      % for icra
%\documentclass[10pt, a4paper]{IEEEtran}      % for ecmr
\documentclass{article}
\usepackage{graphicx}
\usepackage{booktabs}
\usepackage{amsmath}
\usepackage[ruled,lined]{algorithm2e}

%\IEEEoverridecommandlockouts                              % This command is only
%\overrideIEEEmargins


\title{\LARGE \bf
    Title of our awesome paper
}

\author{
Barbora Kozl\'\i kova, Vojt\v{e}ch Von\'{a}sek, Martin Ma\v{n}\'{a}k
}

%Department of Cybernetics, Faculty of  Electrical Engineering\\ Czech Technical University in Prague
%Technicka 2, 166 27, Prague 6, Czech Republic.

\begin{document}

%\maketitle
\thispagestyle{empty}
\pagestyle{empty}

\begin{abstract}
TODO
\end{abstract}

\section{Introduction}



How the AWD are computed
paper of MM \cite{manak2017hybrid}


The ligand trajectories are computed using sampling-based motion planning originally studied in the field of robotics.
The ligand is considered as a robot moving amongst obstacles defined by the atoms of the proteins.
Since the ligand can translate, rotate and also change its dihedral angles, the planning leads to seach a in
a high-dimensional configuration space.
This space is searched using modified Rapidly Exploring Random Tree method~\cite{vonasek2017tunnel}.
The algorithm builds a tree of collision-free configurations (each configuration describes the pose, translation and dihedral angles)
rooted at the initial configuration.
In each iteration, the method incrementally extends the tree by randomized sampling of space along the supplied tunnel.


%\includegraphics[width=0.2\textwidth]{fig/graphisbt}  
%\includegraphics[width=0.25\textwidth]{fig/graph3d} 

\bibliographystyle{plain}
\bibliography{poster}

\end{document}
