\documentclass[conference]{IEEEtran}
\IEEEoverridecommandlockouts
\usepackage{graphicx}
\usepackage{booktabs}
\usepackage{amsmath}
\usepackage[ruled,lined]{algorithm2e}

%\IEEEoverridecommandlockouts                              % This command is only
%\overrideIEEEmargins


\title{Analysis traversability of  protein tunnels considering flexible ligands}

\author{
Barbora Kozl\'\i kova, Vojt\v{e}ch Von\'{a}sek, Martin Ma\v{n}\'{a}k
}

%Department of Cybernetics, Faculty of  Electrical Engineering\\ Czech Technical University in Prague
%Technicka 2, 166 27, Prague 6, Czech Republic.

\begin{document}

\maketitle

\begin{abstract}
TODO
\end{abstract}

\section{Introduction}



How the AWD are computed
paper of MM \cite{manak2017hybrid}

By searching paths in the constructed AWD, tunnels can be identified in the protein.
These tunnels however describe possible pathways for single atom.

To compute trajectories for non-spherical ligands, considering their shape and conformational changes, 
motion planning techniques, originally studied in the field of robotics, are applied.
The ligand is considered as a flexible robot moving among obstacles defined by the protein atoms.
It is necessary to consider ligand translation, rotation and also additional degrees of freedom defined by the flexible dihedral angles.
Finding of trajectories for such a system leads to a search in a high-dimensional configuration space, which can be efficiently
solved using sampling-based planners~\cite{Lav06}.

We utilize a modified Rapidly Exploring Random Tree Planner (RRT)~\cite{vonasek2017tunnel}, that builds
a tree of collision-free configurations of the ligand (a configuration defines position, translation and internal DOFs of the ligand).
Unlike classic RRT-based planners, that sample the space uniformly, we utilize the constructed AWD to sample the position
of the ligands near the vertices of the AWD.
This allows us to find trajectories along the tunnels and it is also faster than sampling the whole configuration space uniformly.
In each iteration, a random sample is generated around a vertex of AWD and its nearest node in the tree is found.
Then, the tree is expanded towards the random sample from the nearest node. 
During the expansion, collision-detection between the ligand and the protein is checked and the tree is expanded only by the 
collision-free nodes.
The method terminates if the tree reaches the desired goal state, which is  defined e.g. as the position of the active site.

The resulting trajectories are then evaluated using an energy function (using Vina Autodock tool~\cite{trott2009autodock}).


..
visualization

%\includegraphics[width=0.2\textwidth]{fig/graphisbt}  
%\includegraphics[width=0.25\textwidth]{fig/graph3d} 

\bibliographystyle{plain}
\bibliography{poster}

\end{document}
